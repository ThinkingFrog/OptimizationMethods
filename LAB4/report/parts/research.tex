\documentclass[../body.tex]{subfiles}
\begin{document}
\subsection{Линии уровней и градиентных ломанных}
\begin{figure}[H]
    \centering
    \includegraphics[scale=0.5]{parts/levels.jpg}
    \caption{Линии уровня функции}
\end{figure}

\begin{figure}[H]
    \centering
    \includegraphics[scale=0.35]{parts/fastestDesc.jpg}
    \caption{Градиентная ломанная для метода наискорейшего спуска}
\end{figure}

\begin{figure}[H]
    \centering
    \includegraphics[scale=0.39]{parts/DFP.jpg}
    \caption{Градиентная ломанная для ДФП-метода}
\end{figure}

\begin{figure}[H]
    \centering
    \includegraphics[scale=0.5]{parts/Newton.jpg}
    \caption{Градиентная ломанная для метода Ньютона}
\end{figure}

\begin{figure}[H]
    \centering
    \includegraphics[scale=0.5]{parts/research1.jpg}
\end{figure}

\begin{figure}[H]
    \centering
    \includegraphics[scale=0.3]{parts/research2.jpg}
\end{figure}

\subsection{Сравнительный анализ метод}
\begin{figure}[H]
    \centering
    \includegraphics[scale=0.3]{parts/research3.jpg}
\end{figure}

\begin{figure}[H]
    \centering
    \includegraphics[scale=0.3]{parts/research4.jpg}
\end{figure}

\begin{figure}[H]
    \centering
    \includegraphics[scale=0.5]{parts/research5.jpg}
\end{figure}

\begin{figure}[H]
    \centering
    \includegraphics[scale=0.35]{parts/research6.jpg}
\end{figure}
\end{document}