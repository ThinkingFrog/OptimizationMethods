\begin{document}
Увеличение объёмов информации в современном мире привело к тому, что ручная обработка такой информации стала невозможной и возникла небходимость в создании систем и алгоритмов, которые автоматизируют эту работу. Одной из задач обработки информации оказалась задача распознавания - определение признаков, которые отличают один набор данных от других.

Обработка неструктурированных данных (фото, видео, аудио) является достаточно сложной, и самым популярным способом решения такой задачи является машинное обучение, которое пытается воссоздать процесс человеческого обучения на компьютере - группировку объектов в классы по некоторым признакам. Выделяются два вида машинного обучения: "с учителем" и "без учителя" [1, 19]:
\begin{itemize}
    \item "Обучение с учителем" предполагает наличие некой исходной выборки, заведомо разделённой на классы учителем, а системе предлагается обнаружить общие признаки, которые будут описывать класс. В дальнейшем, система сможет на основе этих признаков распределять неразмеченные данные по классам. Такое обучение решает задачу классификации, когда количество классов заранее известно, и от системы требуется лишь отнести данные к одному из них [1, 19].
    \item "Обучение без учителя" может предполагать неизвестное число классов, а входные данные изначально не являются размеченными. Система должна сама определить правила, которые различают предложенные объекты, и на их основе создать классы. Классы могут быть как уже известные, так и созданые новые в процессе распознавания. Такое обучение уже решает задачу распознавания - формирование правила, которое разделяет объекты разных классов [1, 19].
\end{itemize}

Самым распространённым видом информации в современном мире является фотография, пользователи социальных сетей активно делятся ими, а качество систем фото- и видеонаблюдения неуклонно растёт. Такое широкое распространение фотографии неуклонно приводит к тому, что задача распознавания ставится и в этом поле - распознавание информации с фотоснимков. Распознавание снимков чаще всего ставит цель в распознавании объектов, будь то буквы, цифры, дома, животные или люди. В этой работе мы рассмотрим частный случай - задачу распознавания лиц со снимков.

Тем не менее, задача распознавания может быть поставлена и в других сферах. Например, также крайне популярным и развивающимся полем является распознавание в речи, которое ставит сразу множество задач, таких как преобразование речи в текст, синтез речи из текста и определение дикторов и относящихся к ним фраз.

Распознавание объектов на изображениях можно разбить на ряд подзадач [1, 20]:
\begin{itemize}
    \item Сопоставление
    \item Поиск
    \item Восстановление
    \item Классификация
\end{itemize}

На примере распознавания лиц эти пункты можно описать как:
\begin{itemize}
    \item Анализ набора изображений для определения принадлежности к одному и тому же классу
    \item Поиск на изображении фрагмента для распознавания
    \item Восстановление пропущенных фрагментов по контексту
    \item Определение класса, к которому относится изображение
\end{itemize}

Для того чтобы распознавать человеческие лица, выделим три основных аттрибута: возраст, расу и пол. Практически такое разделение может быть применено во многих сферах: поисковая выдача, оценка аудитории, реклама, обучение, возрастные и половые ограничения и многие другие [1, 22]. Существуют также и другие признаки, по которым возможно разделение личностей, но в данной работе будут рассматриваться указанные выше атрибуты.

Задача распознавания по этим признакам интересна ещё и тем, что каждый из них имеет разные категории: числовую, бинарную и множественную нечисловую. А задача определения возраста усложняется ещё и тем, что признаки старения у разных людей проявляются по-разному [1, 25].
\end{document}