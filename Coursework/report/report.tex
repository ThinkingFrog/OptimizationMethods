\documentclass[12pt,a4paper]{article}

\usepackage[T1,T2A]{fontenc}
\usepackage[utf8]{inputenc}
\usepackage[english, russian]{babel}
\usepackage{indentfirst}
\usepackage{misccorr}
\usepackage{graphicx}
\usepackage{amsmath}
\usepackage{graphicx}
\usepackage{float}
\usepackage{hyperref}
\usepackage[left=20mm,right=10mm, top=20mm,bottom=20mm,bindingoffset=0mm]{geometry}

\setlength{\parskip}{6pt}
\DeclareGraphicsExtensions{.png}

\begin{document}
\section{Введение}
\section{Постановка задачи}
\subsection{Общая постановка задачи распознавания}
Ссылка 2
\subsection{Постановка задачи по определению атрибутов личности по изображению лица}
Ссылка 1
\section{Классический метод машины опорных векторов}
\subsection{Краткое описание метода}
Ссылка 3
\subsubsection{Вспомогательные методы}
Ссылка 2
\subsubsection{Идея "один против всех"}
\subsection{Идея "один против одного"}
\subsubsection{Метод локальных бинарных шаблонов (LBP)}
\subsubsection{Метод построения гистограммы опорных градиентов (HOG)}
Ссылка 5
\subsubsection{Метод формирования гистограмм направления края
изображения (EOH)}
\subsubsection{Метод активной модели внешности (AAM)}
\subsubsection{Метод активной модели формы (ASM)}
\section{Модификации машины опорных векторов}
\subsection{Метод релевантных векторов (RVM)}
\subsection{1-norm SVM (LASSO SVM)}
\subsection{Doubly Regularized SVM (ElasticNet SVM)}
\subsection{Support Feature Machine (SFM)}
\subsection{Relevance Feature Machine (RFM)}
\subsection{Безпризнаковое распознавание}
\subsection{Использование скалярного произведения}
\subsection{Relational Discriminant Analysis}
\subsection{Обобщающие методы}
Ссылка 6
\subsubsection{Стратегия обучения с фиксированным зазором}
\subsubsection{Стратегия обучения с суммированием зазоров}
\section{Сравнительный анализ методов}
\section{Краткие выводы}
\section{Список литературы}
\begin{enumerate}
    \item \href{https://mpei.ru/diss/Lists/FilesDissertations/369-Диссертация.pdf}{$https://mpei.ru/diss/Lists/FilesDissertations/369-Диссертация.pdf$}
    \item \href{https://lepskiy.ucoz.ru/Posobie/MMPR_.pdf}{$https://lepskiy.ucoz.ru/Posobie/MMPR_.pdf$}
    \item \href{http://www.machinelearning.ru/wiki/images/8/8b/MOTP11_3.pdf}{$http://www.machinelearning.ru/wiki/images/8/8b/MOTP11_3.pdf$}
    \item \href{https://www.intechopen.com/books/advances-in-character-recognition/svm-classifiers-concepts-and-applications-to-character-recognition}{$https://www.intechopen.com/books/advances-in-character-recognition/svm-classifiers-concepts-and-applications-to-character-recognition$}
    \item \href{https://cyberleninka.ru/article/n/raspoznavanie-dorozhnyh-znakov-s-pomoschyu-metoda-opornyh-vektorov-i-gistogramm-orientirovannyh-gradientov}{$https://cyberleninka.ru/article/n/raspoznavanie-dorozhnyh-znakov-s-pomoschyu-metoda-opornyh-vektorov-i-gistogramm-orientirovannyh-gradientov$}
    \item \href{http://www.mmro.ru/files/2007-mmro-13.pdf}{$http://www.mmro.ru/files/2007-mmro-13.pdf$}
\end{enumerate}

\end{document}
