\documentclass[12pt,a4paper]{article}
\begin{document}
\begin{enumerate}
    \item Рыбинцев А.В. Исследование, модификация и разработка методов компьютерного зрения для задач определения атрибутов личности по изображению лица: диссертация на соискание ученой степени кандидата технических наук. - М.: НИУ МЭИ, 2018.
    \item Лепский А.Е., Броневич А.Г. Математические методы распознавания образов: курс лекций. - Таганрог: Технологический институт Южного федерального университета, 2009.
    \item Татарчук А.И. Байесовские методы опорных векторов для обучения распознаванию образов с управляемой селективностью отбора признаков: автореферат диссертации на соискание ученой степени кандидата физико-математических наук. - М.: ФГБУН «Вычислительный центр им. A.A. Дородницына Российской академии наук», 2014.
    \item Математические методы распознавания образов: Доклады 13-й Всероссийской конференции, посвящённой 15-летию РФФИ. - М., 2007.
    \item Воронцов К.В. Машинное обучение: курс лекций [Электронный ресурс]: Линейные методы классификации и регрессии: метод опорных векторов.\\ - URL: \url{http://machinelearning.ru/wiki/images/archive/a/a0/20160310092432!Voron-ML-Lin-SVM.pdf} (дата обращения: 02.05.2021).
    \item Ссылки от Димки: \url{https://scikit-learn.org/stable/modules/generated/sklearn.svm.SVC.html} \url{https://github.com/eriklindernoren/ML-From-Scratch} \url{https://towardsdatascience.com/support-vector-machines-learning-data-science-step-by-step-f2a569d90f76} \url{https://scikit-learn.org/stable/modules/generated/sklearn.datasets.load_digits.html} \url{https://scikit-learn.org/stable/modules/generated/sklearn.metrics.accuracy_score.html} 
\end{enumerate}
\end{document}
