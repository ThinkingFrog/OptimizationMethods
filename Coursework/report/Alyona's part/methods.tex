\documentclass[12pt,a4paper]{article}
\begin{document}
	Рассматривается задача математического программирования
	\begin{equation}\label{eq:0}
		f(x) \longleftrightarrow min, F(x) = 0, G(x) \leq 0
	\end{equation}
, где $f:R^n \longleftrightarrow R$ -гладкая функция, а $F:R^n \longleftrightarrow R^l$ и $G:R^n \longleftrightarrow R^m$ - гладкие отображения.
	- гладкие отображения.
	\subsection{Метод внутренней точки (IP)}
	Метод внутренней точки - это метод позволяющий решать задачи выпуклой оптимизации с условиями, заданными в виде неравенств, сводя исходную задачу к задаче выпуклой оптимизации.
	\subsection{Последовательное квадратичное программирование (SQP)}	
	Последовательное квадратичное программирование (англ. Sequential quadratic\\ programming (SQP)) — один из наиболее распространённых и эффективных оптимизационных алгоритмов общего назначения, основной идеей которого является последовательное решение задач квадратичного программирования, аппроксимирующих данную задачу оптимизации. Для оптимизационных задач без ограничений алгоритм SQP преобразуется в метод Ньютона поиска точки, в которой градиент целевой функции обращается в ноль. Для решения исходной задачи с ограничениями-равенствами метод SQP преобразуется в специальную реализацию ньютоновских методов решения системы Лагранжа.\\
	
	SQP занимает важнейшее место среди ньютоновских методов.\\
	
	Эти методы генерируют траекторию $\left\lbrace x^k \right\rbrace \subset R^n$ следующим образом: по текущему приближению $x^k$  очередное приближение $x^{k+1}$
	ищется как локальное решение
	(или как стационарная точка) задачи квадратичного программирования
	\begin{equation}\label{eq:1}
		\left\langle f'(x^k), x-x^k \right\rangle  + \frac{1}{2}\left\langle H_k(x-x^k), x-x^k \right\rangle \longrightarrow min,
	\end{equation}


	\begin{equation}\label{eq:2}
		F(x^k)+F(x^k)(x-x^k)=0, G(x^k)+G'(x^k)(x-x^k) \leq 0,
	\end{equation}

	где $H_k$ - симметрическая $n \times n$-матрица, которая в некотором смысле аппроксимирует $\frac{\partial^2 L}{\partial x^2}(\bar{x}, \bar{\lambda}, \bar{\mu})$ при $k\longrightarrow\infty$\\ Например, можно полагать
	\begin{equation}
	H_k = \frac{\partial^2 L}{\partial x^2}(x, \lambda, \mu)
	\end{equation}
	если параллельно с прямой траекторией ${x^k}$ генерировать двойственную траекторию ${(\lambda^k, \mu^k)}$,
	например, следующим образом: по текущим $\lambda^k, \mu^k$ очередная пара $(\lambda^{k+1}, \mu^{k+1})$ определяется как пара множителей Лагранжа, отвечающих стационарной точке $x^{k+1}$ задачи \ref{eq:1}, \ref{eq:2}. 

	\subsubsection{Условия локальной сходимости}
	Как было сказано выше, метод SLQ и SLQP являются Ньютоновским методами, условия сходимости диктуются семейством Ньютоновских методов.
	Локальное поведение методов ньютоновского типа для задачи \ref{eq:0} обычно исследуют предполагая выполнение в искомом локальном решении х этой задачи тех или иных условий регулярности ограничений и достаточных условий второго порядка.
	Важнейшим условием регулярности ограничений является условие Мангасариана-Фромовица \textbf{MFCQ (от английского Mangasarian-Fromovitz constraint qualification)}:
	\begin{equation}
		rankF(\bar{x}) = l, \exists\bar\xi \in kerF'(\bar{x} : G'_{l(\bar{x})}(\bar{x})\bar{\xi} \leq 0 
	\end{equation}\label(eq:3)
	Выполнение \textbf{MFCQ} в стационарной точке $\bar{x}$ задачи \ref{eq:0} равносильно ограниченности полиэдра. Единственности множителей Лагранжа это условие, вообще говоря, не гарантирует. Комбинация MFCQ и требование единственности отвечающих $\bar{x}$ множителей Лагранжа $\bar{\lambda}$ и $\bar{\mu}$ называется строгим условием регулярности Мангасариана-Фромовица SMFCQ (от английского Strict
	Mangasarian-Fromovitz constraint qualification). Это условие можно записать в виде:
	\begin{equation}
		\begin{pmatrix}
		F'(x)\\
			G'_{l_{+}(\bar{x}, \bar{\mu})}(\bar{x})
		\end{pmatrix} = l + |I_{+}(\bar{x}, \bar{\mu})|,
	\exists \xi \in kerF'(\bar{x}): 	G'_{l_{+}(\bar{x}, \bar{\mu})}(\bar{x})\bar{xi} = 0, 	G'_{0_{+}(\bar{x}, \bar{\mu})}(\bar{x})\bar{xi} < 0,
	\end{equation}
где 
$I_+(\bar{x},\bar{xi}) = {i \in I(\bar{x})|\bar{\mu}_i > 0}$, 
$ I_0(\bar{x},\bar{xi}) = I(\bar{x}) \ I_+(\bar{x},\bar{xi})$	
	Для $(\bar{\lambda}, \bar{\mu})$ достаточное условие второго порядка SOSC (от английского Second-order
	sufficient condition) имеет вид
	\begin{equation}
		\frac{\partial^2 L}{\partial x^2}(\bar{x}, \bar{\lambda}, \bar{\mu}\xi, \xi) > 0 \forall\xi \in C(\bar{x})	
	\end{equation}	 
	где $C(\bar{x}) = {\xi \in F'(\bar{x})|G'_{I(\bar{x})}(\bar{x})\xi \leq 0)}, <f'(\bar{x}), \xi\leq 0>$ есть критический конус задачи \ref{eq:0} в точке $\bar{x}$
Локальная сходимость методов SQP и SQLP со сверхлинейной скоростью может быть доказана при выполнении SMFCQ и SOSC. Этот результат был получен
в [33] (см. также [29, § 4.5]). При этом предполагается, что в качестве очередного прямодвойственного приближения $(x^{k+1},\lambda^{k+1}, \mu^{k+1})$ берется ближайшее (или, во всяком случае, достаточно
близкое) к $(x^{k},\lambda^{k}, \mu^{k})$ решение системы ККТ задачи \ref{eq:1}, \ref{eq:2} (в данных предположениях решение этой системы может не быть единственным). Последнее требование, конечно, нельзя назвать абсолютно конструктивным, но стремление к его выполнению может быть реализовано в
используемых методах решения итерационных задач квадратичного программирования \ref{eq:1}, \ref{eq:2} .	
	\subsection{Sequential minimal optimization (SMO)}	
\end{document}
