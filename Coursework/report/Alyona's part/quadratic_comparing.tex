\documentclass[12pt,a4paper]{article}
\begin{document}
	\subsection{Методы квадратичного программирования}
	\subsubsection{Метод внутренней точки}
	Преимущества [8, 7]:
	\begin{itemize}
		\item Теоретическая сходимость за \textbf{$O(\sqrt{n}\log(\frac{1}{\epsilon}))$}
		итераций. Эта теоретическая оценка худшего числа итераций является наилучшей на сегодняшний день оценкой
		среди всех методов решения задач условной оптимизации.
		\item На практике метод внутренней точки сходится за
		константное число итераций, практически не зависящее от размерности задачи. 
		\item Отлично подходит для использования в случае данных большого объема.
		\item Позволяет находить решение задачи квадратичного программирования, в том числе, для высокой точности $\epsilon$.
	\end{itemize}
	
	Недостатки [8, 7]:
	\begin{itemize}
		\item Методы внутренней точки требуют нескольких итераций, но итерации дороги: каждая по сути является шагом Ньютона, и каждая должна выполняться с нуля. 
		\item Теряет свои преимущества на небольших задачах 
	\end{itemize}
	
	\subsubsection{Последовательное квадратичное программирование}
	Преимущества [6, 2]:
	\begin{itemize}
		\item Не требует предварительного решения линейной задачи.
		\item Учитывается погрешность, возникшая на предыдущих
		итерациях
		\item Требует меньше времени по сравнению c другимы методами квадратичного программирования
	\end{itemize}
	
	
	Недостатки [6, 3]:
	\begin{itemize}
		\item Не гарантирует, что решение удовлетворяет вашим ограничениям в конце каждой итерации. Он только гарантирует, что оптимальный план удовлетворяет вашим ограничения. Это означает, что если SQP когда-либо не удастся найти оптимальный план, возможно, не будет доступных промежуточных решений, которые будут лучше первоначального приближение.
		\item Дает преимущество, только когда значительное количество
		переменных не меняются в ходе итерации
	\end{itemize}
	
	\subsubsection{Последовательная минимальна оптимизация(SMO)}
	Преимущества [14]:
	\begin{itemize}
		\item Простой алгоритм, который быстро решает проблемы SVM QP без дополнительного хранения матрицы и без привлечения итерационного численного порядка подготовки для каждой позадачи
		\item Решение для двух множителей Лагранжа может быть выполнено
		аналитически
		\item Высокий уровень точности
	\end{itemize}
	
	Недостатком можно назвать, что хотя этот алгоритм гарантирует сходимость, но ускорить сходимость можно только эврестически подбирая параметры.
\end{document}