\documentclass[12pt,a4paper]{article}

\usepackage[T1,T2A]{fontenc}
\usepackage[utf8]{inputenc}
\usepackage[english, russian]{babel}
\usepackage{indentfirst}
\usepackage{misccorr}
\usepackage{graphicx}
\usepackage{amsmath}
\usepackage{graphicx}
\usepackage{float}
\usepackage[left=20mm,right=10mm, top=20mm,bottom=20mm,bindingoffset=0mm]{geometry}

\setlength{\parskip}{6pt}
\DeclareGraphicsExtensions{.png}

\begin{document}
\section{Краткий вывод}
Использование метода опорных векторов является существенно менее трудоемким и дает большую точность классификации при определении атрибута "пол" на неподготовленных изображениях. Наиболее перспективным методом формирования вектора признаков является метод локальных бинарных шаблонов (LBP) как для задачи определения
атрибутов "пол" и "раса", так и для атрибута "возраст". Их сочетание дает точность для атрибута "пол" порядка 87\%, для атрибута "раса" - около 81.3\%, а при прогнозировании возраста личности можно достичь точности в 53\%.

Существующие на сегодня методы классификации изображений лиц по атрибуту "пол" в большинстве случаев позволяют правильно классифицировать примерно две трети реальных изображений, что дает предпосылку к созданию модификаций с целью повышения качества классификации.
\end{document}
