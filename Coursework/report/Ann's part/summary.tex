\documentclass[main.tex]{subfiles}
\begin{document}
Использование метода опорных векторов является существенно менее трудоемким и дает большую точность классификации при определении атрибута "пол" на неподготовленных изображениях. Наиболее перспективным методом формирования вектора признаков является метод локальных бинарных шаблонов (LBP) как для задачи определения атрибутов "пол" и "раса", так и для атрибута "возраст". Их сочетание дает точность для атрибута "пол" порядка 87\%, для атрибута "раса" - около 81.3\%, а при прогнозировании возраста личности можно достичь точности в 53\% [1, 124-128].

Существующие на сегодня методы классификации изображений лиц по атрибуту "пол" в большинстве случаев позволяют правильно классифицировать примерно две трети реальных изображений, что дает предпосылку к созданию модификаций с целью повышения качества классификации.

Касательно методов квадратичного программирования, метод последовательной минимальной оптимизации при использовании машины опорных векторов лучше всего подходит для решения указанной в практической части задачи распознавания. На втором месте по точности среди рассматриваемых методов стоит метод последовательного квадратичного программирования наименьших квадрат, а наименее точным методом оказался метод внутренних точек.
\end{document}
