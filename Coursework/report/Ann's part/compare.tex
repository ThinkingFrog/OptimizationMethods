\documentclass[12pt,a4paper]{article}

\usepackage[T1,T2A]{fontenc}
\usepackage[utf8]{inputenc}
\usepackage[english, russian]{babel}
\usepackage{indentfirst}
\usepackage{misccorr}
\usepackage{graphicx}
\usepackage{amsmath}
\usepackage{graphicx}
\usepackage{float}
\usepackage[left=20mm,right=10mm, top=20mm,bottom=20mm,bindingoffset=0mm]{geometry}

\setlength{\parskip}{6pt}
\DeclareGraphicsExtensions{.png}

\begin{document}
\subsection{Метод опорных векторов (SVM)}
Преимущества SVM:
\begin{itemize}
    \item SVM имеет свойство \textit{разреженности}, то есть можно исключить из рассмотрения нулевые $\lambda_i$ и построить компактный классификатор (рещающую функцию) [1, 60]
    \item Метод имеет модификацию преобразования множественной классификации в последовательность бинарных классификаций
    \item SVM позволяет работать с линейно неразделимыми обучающими выборками
    \item Данный метод показывает один из наилучших на данный момент результатов по точности классификации в сочетании с формированием вектора признаков изображения на основе метода LBP [1, 78]
\end{itemize}

Недостатки SVM:
\begin{itemize}
    \item Классическая версия SVM рассчитана на классификацию только по двум классам объектов
\end{itemize}

\subsection{Модификации метода опорных векторов}

\subsubsection{1-norm SVM (LASSO SVM)}
Преимущества LASSO SVM:
\begin{itemize}
	\item  Отбор признаков c параметром селективности $\mu$:
	чем больше $\mu$, тем меньше признаков останется	
\end{itemize}

Недостатки LASSO SVM:
\begin{itemize}
	\item  LASSO начинает отбрасывать значимые признаки,
	когда ещё не все шумовые отброшены
	\item  Нет эффекта группировки (grouping effect):
	значимые зависимые признаки должны отбираться вместе
	и иметь примерно равные веса $w_j$
\end{itemize}

\subsubsection{Doubly Regularized SVM (ElasticNet SVM)}
Преимущества Doubly Regularized SVM:
\begin{itemize}
	\item  Отбор признаков c параметром селективности $\mu$:
	чем больше $\mu$, тем меньше признаков останется
	\item  Есть эффект группировки
\end{itemize}

Недостатки Doubly Regularized SVM:
\begin{itemize}
	\item  Шумовые признаки также группируются вместе,
	и группы значимых признаков могут отбрасываться,
	когда ещё не все шумовые отброшены
\end{itemize}

\subsubsection{Метод релевантных признаков с регулируемой селективностью}
Преимущества RFM:
\begin{itemize}
	\item Отбор признаков c параметром селективности $\mu$:
	чем больше $\mu$, тем меньше признаков останется
	\item Есть эффект группировки
	\item Лучше отбирает набор значимых признаков, когда
	они только совместно обеспечивают хорошее решени
\end{itemize}

Недостатки RFM:
\begin{itemize}
	\item Достаточно трудоемкий метод, в сравнение с базовым SVM
\end{itemize}

\subsubsection{Метод опорных признаков с регулируемой селективностью}
Преимущества SFM:
\begin{itemize}
	\item Отбор признаков c параметром селективности $\mu$
	\item Есть эффект группировки
	\item Значимые зависимые признаки $(|w_j| > \mu)$ группируются и входят в решение совместно (как в Elastic Net),
	\item Шумовые признаки $(|w_j| < \mu)$ подавляются независимо
	(как в LASSO)
\end{itemize}

Недостатки SFM:
\begin{itemize}
	\item Достаточно трудоемкий метод, в сравнение с базовым SVM
\end{itemize}

\subsubsection{Практическое сравнение модификаций SVM}
В дисертации [3, 19] приводятся результаты экспериментального исследования рассмотренных методов обучения с регулируемой селективностью, а именно, метода релевантных признаков и метода опорных признаков. Основной целью экспериментального исследования является анализ предложенных методов в сравнении с существующими методами SVM, Lasso SVM и Elastic Net SVM по их способности
сокращать признаковое описание объектов распознавания и, в конечном итоге, повышать
обобщающую способность обучения при относительно малой обучающей выборке и
большом числе признаков.
\\


Экспериментальное исследование имеет общепринятую структуру, включающую в себя серию модельных экспериментов и пример решения прикладной задачи. \\
Модельные
эксперименты диссертации основаны, по своей структуре, на исследовании, проведенном
авторами метода Elastic Net SYM для иллюстрации преимущества модульноквадратичной функции штрафа Elastic Net по сравнению с традиционным модульным Lasso. Прикладной задачей является задача распознавания рака легких.
Структура четырех модельных задач настоящей диссертации соответствует
четырем простым требованиям к селективному обучению - (\ref{points})\\ Результаты модельных экспериментов, изложенные в диссертации, наглядно иллюстрируют весьма важный факт, что ни один из существующих методов селективного обучения, включая предлагаемые, не удовлетворяет сразу всем этим весьма неизощренным
требованиям. 
\\

Предложенный метод опорных признаков хорошо справился с подавлением шумовых признаков при линейно независимых информативных признаках, существенно улучшив и без того неплохой результат существующего метода Elastic Net
SVM. Однако метод релевантных признаков в этих условиях оказался далеко не так эффективен. Требование выделять группу информативных признаков, которые только
вместе обеспечивают достаточную точность распознавания, оказалось весьма проблематичным, как для существующего Elastic Net SVM, так и для предложенного метода
опорных признаков, селективность которых обеспечивается за счет использования в целевых функциях критериев обучения штрафа модуля. Вместе с тем требование не доставило никаких сложностей для второго предложенного метода релевантных признаков, селективность которого имеет отличную от привычного модуля природу.
\\

Полученные в рамках диссертационной работы результаты экспериментального исследования иллюстрируют полезность рассмотренных методов.


\subsection{Методы формирования вектора признаков изображения}
\subsubsection{Метод локальных бинарных шаблонов (LBP)}
Преимущества LBP:
\begin{itemize}
    \item Метод наиболее информативен с точки зрения формирования вектора признаков по сравнению с другими методами признакового описания
    \item Локальные бинарные шаблоны инварианты к небольшим изменениям в условиях освещенности и небольшим поворотам классифицируемого изображения, что обуславливает их широкое распространение для решения задач определения таких атрибутов личности, как "пол", "раса" и "возрастная группа" [1, 37]
    \item Важным достоинством меттода LBP является простота реализации LBP, что позволяет использовать его в задачах обработки изображений в реальном времени
    \item LBP позволяет сформировать пространство признаков большой размерности (порядка нескольких тысяч), обеспечивая высокую концентрацию информации об исходном изображении, и создает предпосылки для более точной классификации по возрасту
    \item Метод дает возможность использовать только те шаблоны, которые хранят больше информации о локальных особенностях изображения [1, 11]
\end{itemize}

Недостатки LBP:
\begin{itemize}
    \item При небольшом количестве разбиений изображения или его отсутствии теряется информация об расположении локальных особенностей изображения
    \item При формировании вектора признаков изображения получается пространство большой размерности (более 2000), поэтому необходимо вводить модификации для снижения данной размерности и учитывать симметричность лица и различную информативность отдельных участков изображения лица [1, 46]
    \item Особое значение в рассмотрении метода равномерных шаблонов выявлено чисто эмпирически, поэтому существует необходимость описать более формально данный подход и сформировать унифицированное функциональное представление изображений
\end{itemize}

\subsubsection{Метод построения гистограммы направленных градиентов (HOG)}
Данные методы похожи на метод локальных бинарных шаблонов, поэтому они наследуют часть преимуществ и недостатков последнего.

Недостатки:
\begin{itemize}
    \item Размерность пространства признаков при использовании метода HOG плохо поддается сокращению, а процедура вычисления градиента интенсивности заметно сложнее процедуры формирования LBP [1, 39]
    \item Методы более пригодны для решения задач выделения заданного объекта на изображении, а не для определении атрибутов личности
\end{itemize}

\subsubsection{Метод построения активной модели формы (ASM) и активной модели внешности (AAM)}
Преимущества:
\begin{itemize}
    \item Методы позволяют получать хорошие результаты при определении таких атрибутов личности, как "пол" и "раса", так как именно взаимное расположение важных антропометрических точек совместно с текстурой изображения позволяют определить принадлежность личности к определенному полу и расе [1, 44]
\end{itemize}

Недостатки [1, 45]:
\begin{itemize}
    \item Данный подход формирования вектора по признаку "возраст" может использоваться только для определения возрастной группы, а не для вычисления прогнозируемого возраста, так как расположение важный антропометрических точек и расстояние между ними практически не меняются при изменениях в возрасте на несколько лет
    \item Проблемой данного метода является сложность автоматического выделения на изображении характерных антропометрических точек, так как для их выделения изображение анализируется на уровне текстуры (интенсивности пикселей) ввиду отсутствия другой информации об изображении
\end{itemize}
\end{document}
