\begin{document}
\subsection{Метод опорных векторов (SVM)}
Преимущества SVM:
\begin{itemize}
    \item SVM имеет свойство \textit{разреженности}, то есть можно исключить из рассмотрения нулевые $\lambda_i$ и построить компактный классификатор (рещающую функцию) [1, 60]
    \item Метод имеет модификацию преобразования множественной классификации в последовательность бинарных классификаций
    \item SVM позволяет работать с линейно неразделимыми обучающими выборками
    \item Данный метод показывает один из наилучших на данный момент результатов по точности классификации в сочетании с формированием вектора признаков изображения на основе метода LBP [1, 78]
\end{itemize}

Недостатком SVM является его ориентированность на классификацию по двум классам, которую обобщить на несколько классов достаточно проблематично.

\subsection{Методы формирования вектора признаков изображения}
\subsubsection{Метод локальных бинарных шаблонов (LBP)}
Преимущества LBP:
\begin{itemize}
    \item Метод наиболее информативен с точки зрения формирования вектора признаков по сравнению с другими методами признакового описания
    \item Локальные бинарные шаблоны инварианты к небольшим изменениям в условиях освещенности и небольшим поворотам классифицируемого изображения, что обуславливает их широкое распространение для решения задач определения таких атрибутов личности, как "пол", "раса" и "возрастная группа" [1, 37]
    \item Важным достоинством меттода LBP является простота реализации LBP, что позволяет использовать его в задачах обработки изображений в реальном времени
    \item LBP позволяет сформировать пространство признаков большой размерности (порядка нескольких тысяч), обеспечивая высокую концентрацию информации об исходном изображении, и создает предпосылки для более точной классификации по возрасту
    \item Метод дает возможность использовать только те шаблоны, которые хранят больше информации о локальных особенностях изображения [1, 11]
\end{itemize}

Недостатки LBP:
\begin{itemize}
    \item При небольшом количестве разбиений изображения или его отсутствии теряется информация об расположении локальных особенностей изображения
    \item При формировании вектора признаков изображения получается пространство большой размерности (более 2000), поэтому необходимо вводить модификации для снижения данной размерности и учитывать симметричность лица и различную информативность отдельных участков изображения лица [1, 46]
    \item Особое значение в рассмотрении метода равномерных шаблонов выявлено чисто эмпирически, поэтому существует необходимость описать более формально данный подход и сформировать унифицированное функциональное представление изображений
\end{itemize}

\subsubsection{Метод построения гистограммы направленных градиентов (HOG)}
Данные методы похожи на метод локальных бинарных шаблонов, поэтому они наследуют часть преимуществ и недостатков последнего.

Недостатки:
\begin{itemize}
    \item Размерность пространства признаков при использовании метода HOG плохо поддается сокращению, а процедура вычисления градиента интенсивности заметно сложнее процедуры формирования LBP [1, 39]
    \item Методы более пригодны для решения задач выделения заданного объекта на изображении, а не для определении атрибутов личности
\end{itemize}

\subsubsection{Метод построения активной модели формы (ASM) и активной модели внешности (AAM)}
Преимущества:
\begin{itemize}
    \item Методы позволяют получать хорошие результаты при определении таких атрибутов личности, как "пол" и "раса", так как именно взаимное расположение важных антропометрических точек совместно с текстурой изображения позволяют определить принадлежность личности к определенному полу и расе [1, 44]
\end{itemize}

Недостатки [1, 45]:
\begin{itemize}
    \item Данный подход формирования вектора по признаку "возраст" может использоваться только для определения возрастной группы, а не для вычисления прогнозируемого возраста, так как расположение важный антропометрических точек и расстояние между ними практически не меняются при изменениях в возрасте на несколько лет
    \item Проблемой данного метода является сложность автоматического выделения на изображении характерных антропометрических точек, так как для их выделения изображение анализируется на уровне текстуры (интенсивности пикселей) ввиду отсутствия другой информации об изображении
\end{itemize}
\end{document}
