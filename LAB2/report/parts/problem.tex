\documentclass[../body.tex]{subfiles}
\begin{document}
Пусть имеется n пунктов хранения, в которых сосредоточен однотипный груз, и m пунктов назначения. Также известны:
\begin{itemize}
    \item $a_i$ - количество груза в i-ом пункте хранения
    \item $b_j$ - суточная потребность в j-ом пункте назначения
    \item $c_{ij}$ - стоимость перевозки единицы груза из i-ого в j-ый пункт
\end{itemize}

Необходимо составить план перевозок так, чтобы минимизировать стоимость проекта, то есть $\sum_{i=1}^{n}\sum_{j=1}^{m} c_{ij}x_{ij}\rightarrow min$, где $x_{ij}$ - количество груза, перевезенного из i-ого в j-ый пункт. 

Условие транспортной задачи задано в виде таблицы:
\begin{table}[h]
    \centering
    \begin{tabular}{|c|c|c|c|c|c||c|}
        \hline
        & $b_1$ & $b_2$ & $b_3$ & $b_4$ & $b_5$ & \\\hline
        $a_1$ & 3 & 2 & 7 & 11 & 11 & 19\\\hline
        $a_2$ & 2 & 4 & 5 & 14 & 8 & 5\\\hline
        $a_3$ & 9 & 4 & 7 & 15 & 11 & 21\\\hline
        $a_4$ & 2 & 5 & 1 & 5 & 3 & 9\\\hline
        & 12 & 12 & 11 & 8 & 11 & \\\hline
    \end{tabular}
    \caption{Транспортная задача}
    \label{tab:task}
\end{table}

Необходимо:
\begin{enumerate}
	\item Решить транспортную задачу методом потенциалов с выбором начального приближения методом северо-западного угла.
	\item Решить эту же задачу методом перебора крайних точек и сравнить результаты.
	\item Автоматизировать привидение исходной задачи к закрытому виду.
	\item Описать алгоритм построения цикла пересчета.
	\item Решить задачу с усложнением в виде штрафа за недопоставку груза.
\end{enumerate}

Все необходимые алгоритмы \textit{метода перебора крайних точек} описаны в отчете к лабораторной работе по решению задач линейного программирования симплекс-методом.
\end{document}