\documentclass[../body.tex]{subfiles}
\begin{document}
Для того чтобы найденное решение было оптимальным, необходимо выполнение условия $v_j+u_i \leq c_{ij}, i=\overline{1,n}, j=\overline{1,m}$ для ячеек, не входящих в оптимальный опорный план.
\subsection{Проверка результатов метода потенциалов}
Дано:
\begin{table}[h]
    \centering
    \begin{tabular}{|c|c|c|c|c|c||c||c|}
        \hline
        & $b_1$ & $b_2$ & $b_3$ & $b_4$ & $b_5$ & & $u_i$ \\\hline
        $a_1$ & 7 & 4 & - & 8 & - & 19 & 0\\\hline
        $a_2$ & 5 & - & - & - & - & 5 & -1\\\hline
        $a_3$ & - & 8 & 11 & - & 2 & 21 & 2\\\hline
        $a_4$ & - & - & - & - & 9 & 9 & -6\\\hline
        & 12 & 12 & 11 & 8 & 11 & &\\\hline
        \hline
        $v_j$ & 3 & 2 & 5 & 11 & 9 & &\\\hline
    \end{tabular}
    \caption{Оптимальный опорный план, найденный методом потенциалов}
    \label{tab:potentials}
\end{table}

Матрица стоимости перевозок между пунктами:
$$\begin{pmatrix}
    3 & 2 & 7 & 11 & 11\\
    2 & 4 & 5 & 14 & 8\\
    9 & 4 & 7 & 15 & 11\\
    2 & 5 & 1 & 5 & 3\\
\end{pmatrix}$$

Проверим ячейки, не входящие в найденный опорный план:
\begin{equation}
    \left\{
    \begin{array}{ll}
         v_1 + u_3 = 3 + 2 = 5 \leq 9\\
         v_1 + u_4 = 3 - 6 = -3 \leq 2\\ 
         v_2 + u_2 = 2 - 1 = 1 \leq 4\\
         v_2 + u_4 = 2 - 6 = -4 \leq 5\\
         v_3 + u_1 = 5 + 0 = 5 \leq 7\\
         v_3 + u_2 = 5 - 1 = 4 \leq 5\\
         v_3 + u_4 = 5 - 6 = -1 \leq 1\\
         v_4 + u_2 = 11 - 1 = 10 \leq 14\\
         v_4 + u_3 = 11 + 2 = 13 \leq 15\\
         v_4 + u_4 = 11 - 6 = 5 \leq 5\\
         v_5 + u_1 = 9 + 0 = 9 \leq 11\\
         v_5 + u_2 = 9 - 1 = 8 \leq 8\\
    \end{array}
    \right.
\end{equation}

Таким образом, найденный опорный план действительно является оптимальным.

\subsection{Проверка результатов метода перебора}
Проверим, что найденный опорный план, является так же оптимальным, хоть и не равен плану, найденному методом потенциалов.
\begin{table}[h]
    \centering
    \begin{tabular}{|c|c|c|c|c|c||c||c|}
        \hline
        & $b_1$ & $b_2$ & $b_3$ & $b_4$ & $b_5$ & & $u_i$ \\\hline
        $a_1$ & 7 & 12 & - & - & - & 19 & 0\\\hline
        $a_2$ & 5 & - & - & - & - & 5 & -1\\\hline
        $a_3$ & - & - & 11 & - & 10 & 21 & 2\\\hline
        $a_4$ & - & - & - & 8 & 1 & 9 & -6\\\hline
        & 12 & 12 & 11 & 8 & 11 & &\\\hline
        \hline
        $v_j$ & 3 & 2 & 5 & 11 & 9 & &\\\hline
    \end{tabular}
    \caption{Оптимальный опорный план, найденный методом перебора}
    \label{tab:potentials}
\end{table}

Матрица стоимости перевозок между пунктами:
$$\begin{pmatrix}
    3 & 2 & 7 & 11 & 11\\
    2 & 4 & 5 & 14 & 8\\
    9 & 4 & 7 & 15 & 11\\
    2 & 5 & 1 & 5 & 3\\
\end{pmatrix}$$

Проверим ячейки, не входящие в найденный опорный план:
\begin{equation}
    \left\{
    \begin{array}{ll}
         v_1 + u_3 = 3 + 2 = 5 \leq 9\\
         v_1 + u_4 = 3 - 6 = -3 \leq 2\\ 
         v_2 + u_2 = 2 - 1 = 1 \leq 4\\
         v_2 + u_3 = 2 + 2 = 4 \leq 4\\
         v_2 + u_4 = 2 - 6 = -4 \leq 5\\
         v_3 + u_1 = 5 + 0 = 5 \leq 7\\
         v_3 + u_2 = 5 - 1 = 4 \leq 5\\
         v_3 + u_4 = 5 - 6 = -1 \leq 1\\
         v_4 + u_1 = 11 + 0 = 11 \leq 11\\
         v_4 + u_2 = 11 - 1 = 10 \leq 14\\
         v_4 + u_3 = 11 + 2 = 13 \leq 15\\
         v_5 + u_1 = 9 + 0 = 9 \leq 11\\
         v_5 + u_2 = 9 - 1 = 8 \leq 8\\
    \end{array}
    \right.
\end{equation}

Таким образом, найденный опорный план действительно является оптимальным.

\subsection{Проверка результатов задачи с усложнением}
\begin{table}[h]
    \centering
    \begin{tabular}{|c|c|c|c|c|c||c||c|}
        \hline
        & $b_1$ & $b_2$ & $b_3$ & $b_4$ & $b_5$ & & $u_i$ \\\hline
        $a_1$ & 13 & 4 & - & 2 & - & 19 & 0\\\hline
        $a_2$ & - & - & - & - & 5 & 5 & -1\\\hline
        $a_3$ & - & 10 & 11 & - & - & 21 & 2\\\hline
        $a_4$ & - & - & - & 2 & 7 & 9 & -6\\\hline
        $a_5$ & - & - & - & 6 & - & 9 & -2\\\hline
        & 13 & 14 & 11 & 10 & 12 & &\\\hline
        \hline
        $v_j$ & 3 & 2 & 5 & 11 & 9 & &\\\hline
    \end{tabular}
    \caption{Оптимальный опорный план задачи с усложнением}
\end{table}

Матрица стоимости перевозок между пунктами:
$$\begin{pmatrix}
    3 & 2 & 7 & 11 & 11\\
    2 & 4 & 5 & 14 & 8\\
    9 & 4 & 7 & 15 & 11\\
    2 & 5 & 1 & 5 & 3\\
    9 & 9 & 9 & 9 & 9\\
\end{pmatrix}$$

Проверим ячейки, не входящие в найденный опорный план:
\begin{equation}
    \left\{
    \begin{array}{ll}
        v_1 + u_2 = 3 - 1 = 2 \leq 2\\
        v_1 + u_3 = 3 + 2 = 5 \leq 9\\
        v_1 + u_4 = 3 - 6 = -3 \leq 2\\
        v_1 + u_5 = 3 - 2 = 1 \leq 9\\
        v_2 + u_2 = 2 - 1 = 1 \leq 4\\
        v_2 + u_4 = 2 - 6 = -4 \leq 5\\
        v_2 + u_5 = 2 - 2 = 0 \leq 9\\
        v_3 + u_1 = 5 + 0 = 5 \leq 7\\
        v_3 + u_2 = 5 - 1 = 4 \leq 5\\
        v_3 + u_4 = 5 - 6 = -1 \leq 1\\
        v_3 + u_5 = 5 - 2 = 3 \leq 9\\
        v_4 + u_2 = 11 - 1 = 10 \leq 14\\
        v_4 + u_3 = 11 + 2 = 13 \leq 15\\
        v_5 + u_1 = 9 + 0 = 9 \leq 11\\
        v_5 + u_3 = 9 + 2 = 11 \leq 11\\
        v_5 + u_5 = 9 - 2 = 7 \leq 9\\
    \end{array}
    \right.
\end{equation}

Таким образом, найденный опорный план действительно является оптимальным.
\end{document}