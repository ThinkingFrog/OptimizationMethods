\documentclass[../body.tex]{subfiles}
\begin{document}
\subsection{Алгоритм метода северо-западного угла}
	\textbf{Вход}: массив значений запаса и потребности грузов
	\begin{enumerate}
		\item Двигаемся по таблице $n*m$ - размера с верхнего левого угла, заполняя ее ячейки значениями соответсвующих объемов перевозок, в северо-западном направлении
		\item Находим максимально возможный объем груза для соответсвующей ячейки $(i,j)$ посредством подсчета $\min{\{a_i,b_j\}}$ и заполняем этим значением данную ячейку
		\item Вычитаем найденный объем груза из значений соответсвующих полей запаса и потребности
		\item Если $a_i != 0$, то двигаемся вправо по матрице. Если $b_i != 0$, то двигаемся вниз по матрице. Иначе двигаемся по диагонали
		\item Выполняем шаги 2-4, пока не достигнем правого нижнего угла матрицы
	\end{enumerate}
	
    Этот алгоритм позволяет найти начальный план перевозок, то есть допустимую точку, но этот план не будет оптимальным, так как при рассчете не учитывалась стоимость перевозки.
    
\subsection{Алгоритм проверки опорного плана на вырожденность}
    При нахождении опорных планов, в том числе начального, он может оказаться вырожденным. Так как задача рассматривается в закрытом виде, то требуется $n+m-1$ уравнений для ее решения.
    
    Если матрица имеет заполненных клеток меньше, чем требуется уравнений, то этот опорный план является вырожденным и необходимо добавить фиктивный элемент.

\subsection{Алгоритм метода потенциалов}
    Пусть дана таблица, состоящая из элементов $x_{ij}$, где $i=\overline{1,n},j=\overline{1,m}$. Каждой строчке этой таблицы поставим в соответсвие потенциал $u_i$, а столбцам - $v_j$.
    
    \begin{enumerate}
        \item Выписать соотношение $v_j-u_i=c_{ij}$ для каждой клетки, ввести искуственное ограничение, например, $v_0=0$, и решить СЛАУ, поочередно выражая переменные через друг друга
        \item Вычислить параметр $\alpha_{ij}=v_j-u_i$ для ячеек, которые не входят в опорный план. Если $\alpha_{ij} \leq c_{ij}$, то найденный план оптимальный. Иначе пересчитываем план
        \item Ввести в план перевозок ячейку $(i,j)$ из свободных ячеек, для которой $\max{\{\alpha_{ij}-c_{ij}\}}$
        \item Построить цикл пересчета в выбранной ячейке $(i,j)$
        \item В полученном после пересчета списке ячеек найти минимальное значение объема перевозки в ячейках, помеченных знаком минус
        \item Обойти все элементы найденного цикла, применяя к текущей ячейке операцию сложения или вычитания с найденым минимумом объема перевозки в зависимости от значения знака в ячейке
        \item Продолжаем алгоритм с шага 1 после изменения опорного плана
    \end{enumerate}

\subsection{Алгоритм поиска цикла пересчета}	
    Цикл пересчета - последовательность попеременных горизонтальных и вертикальных перемещений в таблице, начиная и заканчивая в выбранной ячейке $(i,j)$. В заполненных ячейках таблицы происходит смена направления движения.
    \begin{enumerate}
        \item Проверить, посещали ли заданную ячейку ранее. Если нет, то пометить ее, чередуя знаки
        \item Проверить тип ячейки:
            \begin{itemize}
                \item Если ячейка не базисная, то продолжаем движение в текущем направлении и запускаем процедуру для следующей клетки
                \item Если ячейка базисная, то запускаем процедуру для всех ячеек, которые доступны по различным направлениям
                \item Если ячейка начальная, то алгоритм прекращает работу, возвращая список ячеек, в которых произошла смена направления
            \end{itemize}
    \end{enumerate}
    
\subsection{Алгоритм приведения задачи к закрытому виду}
\begin{enumerate}
    \item Вычислить общую сумму запасов и общую сумму потребностей
    \item Добавить столбец или строку с нулевой стоимостью перевозки:
    \begin{itemize}
        \item Если превышают запасы, то добавить фиктивного потребителя
        \item Если превышают потребности, то добавить фиктивного поставщика
    \end{itemize}
\end{enumerate}
\end{document}