\documentclass[../body.tex]{subfiles}
\begin{document}
Алгоритм симплекс-метода применим к задачам линейного программирования на нахождение минимума. Метод работает на задачах в канонической форме при всяких вещественных значениях компонент $A\in\mathbb{R}_{m\times{n}},b\in\mathbb{R}_m,c\in\mathbb{R}_n$. Матрица $A$ должна иметь ранг $m$, что гарантирует наличие хотя бы одного опорного вектора.

Поставленная задача является задачей линейного программирования, так как ее целевая функция и ограничения имеют линейный вид. Ранг рассматриваемой матрицы был посчитан встроенным в библиотеку методом $numpy.linalg.matrix\_rank$ и равен 11 так же, как и количество строк данной матрицы. Алгоритмы метода приведения задачи к каноническому виду и симплекс-метода описаны в соответсвующей лабораторной работе и применимы к данной задаче.
\end{document}