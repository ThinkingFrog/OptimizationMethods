\documentclass[../body.tex]{subfiles}
\begin{document}
Даны заготовки с длинной $L=11.7$. Необходимо произвести из заготовок 11 видов изделий в количестве, заданном табл. 1, минимизируя отходы. При этом два изделия имеют одинаковую длину.
\begin{table}[H]
    \centering
    \begin{tabular}{|c|c|c|}
        \hline
        Наименование $(i)$ & Длина $(b_i)$ & Количество $(n_i)$\\\hline
        1 & 0.6 & 249\\\hline
        2 & 0.68 & 60\\\hline
        3 & 0.83 & 97\\\hline
        4 & 1.61 & 76\\\hline
        5 & 1.67 & 72\\\hline
        6 & 1.79 & 18\\\hline
        7 & 2.8 & 43\\\hline
        8 & 3.25 & 5424\\\hline
        9 & 3.25 & 450\\\hline
        10 & 3.7 & 515\\\hline
        11 & 3.95 & 28\\\hline
    \end{tabular}
    \caption{Параметры производства изделий}
\end{table}

\subsection{Формализация задачи}
Для изготовления изделий из заготовок необходимо разрезать последние на части с соответсвующими длинами изделий. Для этих целей составляется таблица $A$ вариантов разреза одной заготовки. В нашем случае это сделано программно перебором соответсвующих коэффициентов $c_i$ в линейной комбинации $plan=\sum_{i=1}^{11}{c_i*b_i}$ при условии $plan\leq L$, где $c_i=\overline{0,\frac{L}{b_i}}$.

Пусть $k_j\in\mathbb{N}$ - необходимое количество заготовок для реализации варианта разреза $j$. Общее количество заготовок $k=\sum_j{k_j}$.

Для решения данной задачи используется симплекс-метод. Функция цели - минимум всех отходов после разрезов $\min\sum_j{k_j*(L-plan)}$. Ограничения задаются транспонированием таблицы $A$ вариантов разреза и вектором свободных членов в виде $\{n_i\}_{i=\overline{1,11}}$, то есть $A^T*K=B$, где $K,B$ - вектора количества заготовок и требуемого количества изделий соответсвенно.

\subsection{Минимизация заготовок и отходов}
Для решения данной задачи используется учет всевозможных вариантов разреза, даже совершенно не оптимальных, так как при подборе оптимального варианта выполнения плана на неоптимальных вариантах разреза должно стоять минимальное или нулевое количество их повторений. То есть данных подход совершенно не влияет на оптимальное решение задачи.

В рамках данной задачи могут существовать две эквивалентные постановки: минимизация количества отходов и минимизация числа необходимых заготовок. При этом для первого случая оптимальным будет такой план, при котором все ограничения выполняются в виде неравенств ввиду, то есть $A^T*K=B$ в указанных выше переменных. А для задачи минимизации числа заготовок можно использовать ограничения вида $A^T*K\geq B$.

\subsection{Формализация задачи для постановки с двойным количеством заготовок для создания изделий}
Если рассматривать ситуацию, что каждое изделение состоит из двух частей одинаковой длины, то к указанной постановке задачи стоит добавить, что вектор свободных членов $B$ домножается покомпонентно на 2. Затем задача решается аналогично.
\end{document}