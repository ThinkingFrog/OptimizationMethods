\documentclass[../body.tex]{subfiles}
\begin{document}
\subsection{Задача с учетом двух равнодлинных изделий}
При решении данной задачи было полученно минимальное количество заготовок, необходимых для выполнения плана, равное $k=2153$. Всего вариантов разреза было получено 26470.
\begin{table}[H]
    \centering
    \begin{tabular}{|c|c|c|}
        \hline
        План разреза & Используемое количество\\\hline
        (0, 0, 0, 0, 0, 0, 3, 1, 0, 0, 0) & 15\\\hline
        (0, 0, 0, 0, 0, 0, 0, 3, 0, 0, 0) & 1486\\\hline
        (3, 0, 0, 0, 0, 0, 0, 3, 0, 0, 0) & 25\\\hline
        (2, 1, 0, 0, 0, 0, 0, 3, 0, 0, 0) & 60\\\hline
        (0, 0, 2, 0, 0, 0, 0, 3, 0, 0, 0) & 49\\\hline
        (0, 0, 0, 1, 0, 0, 0, 3, 0, 0, 0) & 77\\\hline
        (0, 0, 0, 0, 1, 0, 0, 3, 0, 0, 0) & 73\\\hline
        (0, 0, 0, 0, 0, 1, 0, 3, 0, 0, 0) & 18\\\hline
        (0, 0, 0, 0, 0, 0, 0, 0, 3, 0, 0) & 150\\\hline
        (0, 0, 0, 0, 0, 0, 0, 0, 0, 3, 0) & 172\\\hline
        (2, 0, 0, 0, 1, 0, 0, 2, 0, 0, 1) & 28\\\hline
    \end{tabular}
    \caption{Оптимальные варианты разрезов}
\end{table}

\subsection{Задача с учетом включения в изделие двух деталей равной длины}
При решении данной задачи было полученно минимальное количество заготовок, необходимых для выполнения плана, равное $k=4302$. Всего вариантов разреза было получено 26470.
\begin{table}[H]
    \centering
    \begin{tabular}{|c|c|c|}
        \hline
        План разреза & Используемое количество\\\hline
        (0, 0, 0, 0, 0, 0, 3, 1, 0, 0, 0) & 29\\\hline
        (0, 0, 0, 0, 0, 0, 0, 3, 0, 0, 0) & 2972\\\hline
        (3, 0, 0, 0, 0, 0, 0, 3, 0, 0, 0) & 49\\\hline
        (2, 1, 0, 0, 0, 0, 0, 3, 0, 0, 0) & 120\\\hline
        (0, 0, 2, 0, 0, 0, 0, 3, 0, 0, 0) & 98\\\hline
        (0, 0, 0, 1, 0, 0, 0, 3, 0, 0, 0) & 153\\\hline
        (0, 0, 0, 0, 1, 0, 0, 3, 0, 0, 0) & 145\\\hline
        (0, 0, 0, 0, 0, 1, 0, 3, 0, 0, 0) & 36\\\hline
        (0, 0, 0, 0, 0, 0, 0, 0, 3, 0, 0) & 300\\\hline
        (0, 0, 0, 0, 0, 0, 0, 0, 0, 3, 0) & 344\\\hline
        (2, 0, 0, 0, 1, 0, 0, 2, 0, 0, 1) & 56\\\hline
    \end{tabular}
    \caption{Оптимальные варианты разрезов}
\end{table}

\end{document}