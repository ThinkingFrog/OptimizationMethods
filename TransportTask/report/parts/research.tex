\documentclass[../body.tex]{subfiles}
\begin{document}
Сформулируем задачу с усложнением. Пусть за недопоставку некоторого уровня груза начисляется соответсвующий штраф. 
\begin{table}[h]
    \centering
    \begin{tabular}{|c||c|c|c|}
        \hline
        Уровень & 1 & 5 & 10\\\hline
        Штраф & 3 & 9 & 20\\\hline
    \end{tabular}
    \caption{Штраф за недопоставку грузка}
    \label{tab:penalty}
\end{table}

Создадим ситуацию недопоставки, то есть увеличим потребности на несколько единиц. В таком случае сумма всех потребностей будет превышать сумму всех запасов, то есть задача имеет открытый вид.

\begin{table}[h]
    \centering
    \begin{tabular}{|c|c|c|c|c|c||c|}
        \hline
        & $b_1$ & $b_2$ & $b_3$ & $b_4$ & $b_5$ & \\\hline
        $a_1$ & 3 & 2 & 7 & 11 & 11 & 19\\\hline
        $a_2$ & 2 & 4 & 5 & 14 & 8 & 5\\\hline
        $a_3$ & 9 & 4 & 7 & 15 & 11 & 21\\\hline
        $a_4$ & 2 & 5 & 1 & 5 & 3 & 9\\\hline
        & 13 & 14 & 11 & 10 & 12 & \\\hline
    \end{tabular}
    \caption{Транспортная задача с усложнением}
    \label{tab:difficult_task}
\end{table}

В таком случае для использования метода потенциалов необходимо приведение задачи к закрытому виду, при этом в зависимости от значения разности потребностей и запасов вводится фиктивная величина, имеющая стоимость, соответсвующую уровню вычисленной разности. 
\end{document}