\documentclass[../body.tex]{subfiles}
\begin{document}
Метод северо-западного угла для нахождения начального плана не позволяет найти оптимальный план перевозки, так как при заполнении клеток матрицы не учитывается стоимость перевозок груза. Но этот метод имеет достаточно простой алгоритм и позволяет найти начальное приблежение искомого плана для последующего его улучшения.

Метод потенциалов реализует упрощенную процедуру симплекс-метода. Он гарантирует получение оптимального решения, однако является достаточно сложным и требующим больших временных затрат.

К тому же метод потенциалов позволяет задавать условие задачи в более компактном виде, нежели метод перебора крайних точек. Например, исходя из заданного условия, в методе потенциалов мы оперировали матрицей, размерностью 5 на 6, учитывая значения запасов и потребностей, а для метода перебора потребовалось задать матрицу, размерностью 8 на 20, что значительно больше.
\end{document}