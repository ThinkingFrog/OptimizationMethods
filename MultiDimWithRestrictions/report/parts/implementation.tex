\documentclass[../body.tex]{subfiles}
\begin{document}
\subsection{Метод всевозможных направлений Зойтендейка}
\subsubsection{Начальный этап}
\begin{enumerate}
    \item Выбрать совокупность параметров $\{\xi_i\}_0^m:\xi_i>0$, использующихся для улучшения свойств сходимости задачи. В решении рекомендовано взять $\xi_i=1$ для $\forall i=\overline{0,m}$
    \item Выбрать параметр дробления $0<\lambda<1$. Рекомендовано взять для решения задачи $\lambda=\frac{1}{2}$
    \item Выбрать начальное приближение $x_0\in S$, где $S=\{X|\phi_i(x)\leq0$ для $\forall i=\overline{1,m}\}$ и параметр $\eta_0$
    \item Взять критерий близости к почти активным ограничениям $\delta_0=-\eta_0$, где $\delta_0>0$, $J_{\delta_k}(x_k)=\{i\in M|-\delta_k\leq\phi_i(x_k)\leq0\}$ - множество номеров почти активных ограничений для $M=\overline{1,m}$
\end{enumerate}

\subsubsection{Поиск начального приближения}
\begin{enumerate}
    \item Найти $min\eta<0$ при условии $\phi_i(x)*\eta\leq0$, где $i=\overline{1,m}$
    \item Решением будет точка $x_0:\phi_i(x_0)\leq0$, которая является допустимой точкой для исходной задачи
    \item $\eta_0=min\eta$
\end{enumerate}

\subsubsection{Основной этап}
\begin{enumerate}
    \item Известны $x_k\in S$ и $\delta_k>0$
    \item Решить вспомогательную задачу линейного программирования симплекс-методом для определения направления спуска $s:min\eta$ при условиях
        \begin{equation}
            \left\{
            \begin{array}{ll}
                \bigtriangledown^T\phi_0(x_k)*s\leq\eta\xi_0\\
                \bigtriangledown^T\phi_i(x_k)*s\leq\eta\xi_i
            \end{array}
            \right.
        \end{equation}
        для $i\in J_{\delta_k}(x_k)$
    \item Обозначить найденные $s_{\delta_k}(x_k)=s_k$ и $\eta_{\delta_k}(x_k)=\eta_k$
    \item Если $\eta_k<-\delta_k:$
        \begin{itemize}
            \item Делаем шаг $\alpha_k$ по выбранному направлению $x_{k+1}=x_k+\alpha_ks_k$ и $\delta_{k+1}=\delta_k$
            \item Иначе шаг не делается, то есть $x_{k+1}=x_k$ и $\delta_{k+1}=\lambda\delta_k$
        \end{itemize}
    \item Закончить работу алгоритма при $\delta_k<\delta_{0k}$, где $-\delta_{0k}=max\phi_i(x_k)$ для $i\not\in J_0(x_k)$ - не для активных ограничений, и $\eta_k=0$
\end{enumerate}

\subsubsection{Выбор величины шага по принципу дробления}
\begin{enumerate}
    \item Положить $\alpha_k=\alpha_0*\lambda^{ik}$, где $\alpha_0=1$
    \item Выбрать $\alpha_k$, удовлетворяющее условиям:
        \begin{equation}
            \left\{
            \begin{array}{ll}
                \phi_0(x_k+\alpha_k*s_k)-\phi_0(x_k)\leq\xi_0\eta_k\alpha_k\\
                \phi_i(x_k+\alpha_k*s_k)\leq0,i=\overline{1,m}
            \end{array}
            \right.
        \end{equation}
\end{enumerate}
\end{document}