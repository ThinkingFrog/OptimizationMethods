\documentclass[../body.tex]{subfiles}
\begin{document}		
	Симплекс метод был предложен американским математиком Р.Данцигом в 1947 году, с тех пор не утратил свою акиуальность, для нужд промышленности этим методом нередко решаются задачи линейного программирования с тысячами переменных и ограничений. \\
	\vspace{\baselineskip}
	
	Основные преимущества метода:
	\begin{itemize}
		\item Симплекс-метод является универсальным методом, которым можно решить любую задачу линейного программирования, в то время, как графический метод пригоден лишь для системы ограничений с двумя переменными.
		\item Решение будет гарантировано найдено за $O(2^n)$ операций, где n - это количество переменных.
		\item Не так хорош для больших задач, но есть множетсво улучшений базового симплекс-метода, которые компенсируют эту проблему.
	\end{itemize}
	
	
\end{document}