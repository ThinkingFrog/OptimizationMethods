\documentclass[../body.tex]{subfiles}
\begin{document}
	Алгоритм \textbf{симплекс-метода} применим к задачам линейного программирования на нахождение минимума. Метод работает на задачах в канонической форме при всяких вещественных значениях компонент $ A \in \mathbb{R}_{m\times{n}}, b \in \mathbb{R}_{m}, c \in \mathbb{R}_n $.
	Матрица $A$ должно иметь ранг $m$, что гарантирует наличие хотя бы одного опорного вектора.	
	\vspace{\baselineskip}
	\\Проверим применимость \textbf{симплекс-метода} к нашей выбранной задаче. Для вычисления ранга приведем матрицу к ступенчатому виду, используя элементарные преобразования над строками и столбцами матрицы: 
	\begin{multline}
		\begin{pmatrix}
			1 & 2 & 3 & 4 & 0 \\
			2 & 3 & 8 & 0 & 1 \\
			1 & 4 & 0 & 5 & 1 \\
			3 & 7 & 4 & 0 & 2 \\
			2 & 3 & 5 & 6 & 1 \\
		\end{pmatrix}
	\Longrightarrow
		\begin{pmatrix}
			1 & 2 & 3 & 4 & 0 \\
			0 & -1 & 2 & -8 & 1 \\
			0 & 2 & -3 & 1 & 1 \\
			0 & 1 & -5 & -12 & 2 \\
			0 & -1 & -1 & -2 & 1 \\
		\end{pmatrix}
	\Longrightarrow
	\begin{pmatrix}
		1 & 2 & 3 & 4 & 0 \\
		0 & 1 & -2 & 8 & -1 \\
		0 & 0 & 1 & -15 & 3 \\
		0 & 0 & -3 & -20 & 3 \\
		0 & 0 & -3 & 6 & 0 \\
	\end{pmatrix}
	\Longrightarrow \\
\begin{pmatrix}
	1 & 2 & 3 & 4 & 0 \\
	0 & 1 & -2 & 8 & -1 \\
	0 & 0 & 1 & -15 & 3 \\
	0 & 0 & 0 & -65 & 12 \\
	0 & 0 & 0 & -39 & 9 \\
\end{pmatrix}
\Longrightarrow
\begin{pmatrix}
	1 & 2 & 3 & 4 & 0 \\
	0 & 1 & -2 & 8 & -1 \\
	0 & 0 & 1 & -15 & 3 \\
	0 & 0 & 0 & 1 & -\frac{12}{65} \\
	0 & 0 & 0 & 0 & -\frac{9}{39} \\
\end{pmatrix}
	\end{multline}\\
\vspace{\baselineskip}
Так как ненулевых строк 5, то $rang(A) = 5$, столько же, сколько строк в матрице.\\
\vspace{\baselineskip}
Можно сделать вывод, что симплекс-метод применим к нашей задаче.
\end{document}